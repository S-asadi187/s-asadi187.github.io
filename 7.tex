\documentclass[xcolor=dvipsnames,professionalfonts]{beamer}
\author{صديقه اسدی}
\title{فصل7:  مديريت ريسک در توسعه نرم افزار}
\institute{دانشگاه پیام نور واحد تهران شمال}
\newcommand{\abs}[1]{\vert #1\vert}

\usepackage{xcolor}
\usefonttheme{serif}
\usepackage{tikz}
\usepackage{xcolor}
\usefonttheme{serif}

\usepackage{graphicx}
\usetheme{Warsaw}
%\usetheme{AnnArbor}

% در این قسمت می‌توانید رنگ متن را تغییر دهید
\setbeamercolor{normal text}{fg=black}\usebeamercolor*{normal text}

\usecolortheme[named=brown]{structure}
\useoutertheme{infolines}

%%از بخش زیر برای تغییر زمینه اسلاید ستفاده کنید.                          
%\usebackgroundtemplate%
%{%
%	\includegraphics[width=\paperwidth,height=\paperheight]{4.jpg}%
%}

%\usepackage{tikz}

\usepackage{xepersian}
\settextfont{B Roya}
\setlatintextfont{Arial}
%\setdigitfont{B Roya}

%\newtheorem{thm}{قضیه}
%\newtheorem{Def}{تعریف}




%%%%%%%%%%%%%
\raggedleft
\makeatletter
\define@key{beamercolbox}{left}[0pt]{\def\beamer@colbox@rs{0pt}\def\beamer@colbox@ls{#1 plus1fill}}
\makeatother
\raggedleft

\makeatletter
\expandafter\let\csname beamer@@tmpop@itemize item@default\endcsname\relax
\expandafter\let\csname beamer@@tmpop@itemize subitem@default\endcsname\relax
\expandafter\let\csname beamer@@tmpop@itemize subsubitem@default\endcsname\relax

\defbeamertemplate*{itemize item}{default}{\scriptsize\raise1.25pt\hbox{\donotcoloroutermaths$\blacktriangleleft$}}
\defbeamertemplate*{itemize subitem}{default}{\tiny\raise1.5pt\hbox{\donotcoloroutermaths$\blacktriangleleft$}}
\defbeamertemplate*{itemize subsubitem}{default}{\tiny\raise1.5pt\hbox{\donotcoloroutermaths$\blacktriangleleft$}}

\bidi@patchcmd{\@listi}{\leftmargin}{\rightmargin}{}{}
\let\@listI\@listi
\bidi@patchcmd{\@listii}{\leftmargin}{\rightmargin}{}{}
\bidi@patchcmd{\@listiii}{\leftmargin}{\rightmargin}{}{}
\bidi@patchcmd{\beamer@enum@}{\raggedright}{\raggedleft}{}{}
\bidi@patchcmd{\@@description}{\raggedright}{\raggedleft}{}{}
\bidi@patchcmd{\@@description}{\leftmargin}{\rightmargin}{}{}

\renewcommand{\itemize}[1][]{%
\beamer@ifempty{#1}{}{\def\beamer@defaultospec{#1}}%
\ifnum \@itemdepth >2\relax\@toodeep\else
\advance\@itemdepth\@ne
\beamer@computepref\@itemdepth% sets \beameritemnestingprefix
\usebeamerfont{itemize/enumerate \beameritemnestingprefix body}%
\usebeamercolor[fg]{itemize/enumerate \beameritemnestingprefix body}%
\usebeamertemplate{itemize/enumerate \beameritemnestingprefix body begin}%
\list
{\usebeamertemplate{itemize \beameritemnestingprefix item}}
{\def\makelabel##1{%
{%
\hss\llap{{%
\usebeamerfont*{itemize \beameritemnestingprefix item}%
\usebeamercolor[fg]{itemize \beameritemnestingprefix item}##1}}%
}%
}%
}
\fi%
\beamer@cramped%
\raggedleft%
\beamer@firstlineitemizeunskip%
}
\makeatother
\raggedleft

%%%%%%%%%%%%%%%%%%%%%%%%%%%%%%%%%%%

\begin{document}

	\frame{\maketitle}
	\begin{frame}
	
		\frametitle{اهداف یادگیری :}
      \begin{itemize}
     	\item    فرآیند مدیریت ریسک
     	\item    ریسک های توسعه نرم افزار   
     	\item    استراتژی های مقابله با ریسک
     	\item    برنامه ریزی برای مقابله با ریسک
     \end{itemize}
	\end{frame}
\begin{frame}
	
	\frametitle{ریسک}
	%\framesubtitle{subtitle}
	
	
	
	
	
	\begin{itemize}
	
	    	\item ریسک را می توان به عنوان یک تابع از احتمال تهدید یا منشاء تهدید در نظر گرفت  که به نوعی آسیب پذیری و تاثیرات مخرب تهدید را در پروژه یا سازمان مشخص میکند. \lr{(NIST)}
	    	\item در استاندارد \lr{ISO} ریسک نرم افزار به عنوان ترکیبی از احتمالات یک واقعه و تاثیرات آن تعریف می گردد.
	    	\item 
		تاثیرت منفی ریسک :
		\begin{itemize}
			\item ایجاد هزینه های پیش بینی نشده
			\item 		از دست دادن زمان یا تأخیر در پروژه
			\item 		کاهش کارایی و کیفیت نرم‌افزار 
		\end{itemize}
		
 

		\item ریسک ها را نمی توان از پروژه‌های نرم‌افزاری حذف کرد ولی می‌توان آنها را مدیریت یا کنترل کرد
		مدیریت ریسک در مراحل اولیه توسعه نرم‌افزار آغاز می گردد )حتی قبل از امضای قرارداد به منظور به دست آوردن دید کلی از مخاطرات پروژه ( و به عنوان یک فعالیت مستمر درخلال توسعه نرم‌افزار ادامه می‌یابد
		\item خروجی فرآیند تجزیه و تحلیل ریسک عمدتاً به تصمیمات مهم مدیریتی منجر می شود و حتی در بعضی از موارد عامل تعیین کننده در ورود یا عدم ورود به بعضی از پروژه ها می باشد
		
	\end{itemize}
\end{frame}
\begin{frame}
	\frametitle{اسلاد۳}
	%\framesubtitle{subtitle}
	منتنت تناgfgfgf h منن
	\begin{itemize}
		\item ضص
		\item بلذر
	\end{itemize}
\end{frame}

\end{document}