\documentclass[xcolor=dvipsnames,professionalfonts]{beamer}
\author{صديقه اسدی}
\title{فصل7:  مديريت ريسک در توسعه نرم افزار}
\institute{دانشگاه پیام نور واحد تهران شمال}
\newcommand{\abs}[1]{\vert #1\vert}

\usepackage{xcolor}
\usefonttheme{serif}
\usepackage{tikz}

\usepackage{graphicx}
\usetheme{Warsaw}
%\usetheme{AnnArbor}

% در این قسمت می‌توانید رنگ متن را تغییر دهید
\setbeamercolor{normal text}{fg=black}\usebeamercolor*{normal text}

\usecolortheme[named=blue]{structure}


%%از بخش زیر برای تغییر زمینه اسلاید ستفاده کنید.                          
%\usebackgroundtemplate%
%{%
%	\includegraphics[width=\paperwidth,height=\paperheight]{1.jpg}%
%}
%\useoutertheme{infolines}


%\usepackage{tikz}

\usepackage{xepersian}
\settextfont{B Roya}
\setlatintextfont{Arial}
%\setdigitfont{B Roya}

%\newtheorem{thm}{قضیه}
%\newtheorem{Def}{تعریف}




%%%%%%%%%%%%%
\raggedleft
\makeatletter
\define@key{beamercolbox}{left}[0pt]{\def\beamer@colbox@rs{0pt}\def\beamer@colbox@ls{#1 plus1fill}}
\makeatother
\raggedleft

\makeatletter
\expandafter\let\csname beamer@@tmpop@itemize item@default\endcsname\relax
\expandafter\let\csname beamer@@tmpop@itemize subitem@default\endcsname\relax
\expandafter\let\csname beamer@@tmpop@itemize subsubitem@default\endcsname\relax

\defbeamertemplate*{itemize item}{default}{\scriptsize\raise1.25pt\hbox{\donotcoloroutermaths$\blacktriangleleft$}}
\defbeamertemplate*{itemize subitem}{default}{\tiny\raise1.5pt\hbox{\donotcoloroutermaths$\blacktriangleleft$}}
\defbeamertemplate*{itemize subsubitem}{default}{\tiny\raise1.5pt\hbox{\donotcoloroutermaths$\blacktriangleleft$}}

\bidi@patchcmd{\@listi}{\leftmargin}{\rightmargin}{}{}
\let\@listI\@listi
\bidi@patchcmd{\@listii}{\leftmargin}{\rightmargin}{}{}
\bidi@patchcmd{\@listiii}{\leftmargin}{\rightmargin}{}{}
\bidi@patchcmd{\beamer@enum@}{\raggedright}{\raggedleft}{}{}
\bidi@patchcmd{\@@description}{\raggedright}{\raggedleft}{}{}
\bidi@patchcmd{\@@description}{\leftmargin}{\rightmargin}{}{}

\renewcommand{\itemize}[1][]{%
\beamer@ifempty{#1}{}{\def\beamer@defaultospec{#1}}%
\ifnum \@itemdepth >2\relax\@toodeep\else
\advance\@itemdepth\@ne
\beamer@computepref\@itemdepth% sets \beameritemnestingprefix
\usebeamerfont{itemize/enumerate \beameritemnestingprefix body}%
\usebeamercolor[fg]{itemize/enumerate \beameritemnestingprefix body}%
\usebeamertemplate{itemize/enumerate \beameritemnestingprefix body begin}%
\list
{\usebeamertemplate{itemize \beameritemnestingprefix item}}
{\def\makelabel##1{%
{%
\hss\llap{{%
\usebeamerfont*{itemize \beameritemnestingprefix item}%
\usebeamercolor[fg]{itemize \beameritemnestingprefix item}##1}}%
}%
}%
}
\fi%
\beamer@cramped%
\raggedleft%
\beamer@firstlineitemizeunskip%
}
\makeatother
\raggedleft

%%%%%%%%%%%%%%%%%%%%%%%%%%%%%%%%%%%

\begin{document}

	\frame{\maketitle}
	\begin{frame}
	
		\frametitle{اهداف یادگیری :}
	
      \begin{itemize}
     
      
      	
     	\item فرآیند مدیریت ریسک 
     	\item ریسک های توسعه نرم افزار   
     	\item استراتژی های مقابله با ریسک
     	\item برنامه ریزی برای مقابله با ریسک
     \end{itemize}
	\end{frame}
\begin{frame}
	
	\frametitle{ریسک}
	%\framesubtitle{subtitle}
				
	\begin{itemize}
	
	    	\item ریسک را می توان به عنوان یک تابع از احتمال تهدید یا منشاء تهدید در نظر گرفت  که به نوعی آسیب پذیری و تاثیرات مخرب تهدید را در پروژه یا سازمان مشخص میکند. \lr{(NIST)}
	    	\item در استاندارد \lr{ISO} ریسک نرم افزار به عنوان ترکیبی از احتمالات یک واقعه و تاثیرات آن تعریف می گردد.
	    	\item 
		تاثیرت منفی ریسک :
		\begin{itemize}
			\item ایجاد هزینه های پیش بینی نشده
			\item 		از دست دادن زمان یا تأخیر در پروژه
			\item 		کاهش کارایی و کیفیت نرم‌افزار 
		\end{itemize}
		\item ریسک ها را نمی توان  حذف کرد ولی می‌توان آنها را مدیریت یا کنترل کرد
		مدیریت ریسک در مراحل اولیه توسعه نرم‌افزار آغاز می گردد )حتی قبل از امضای قرارداد به منظور به دست آوردن دید کلی از مخاطرات پروژه ( و به عنوان یک فعالیت مستمر درخلال توسعه نرم‌افزار ادامه می‌یابد
		\item خروجی فرآیند تجزیه و تحلیل ریسک  به تصمیمات مهم مدیریتی منجر می شود و حتی در بعضی از موارد عامل ورود یا عدم ورود به بعضی از پروژه ها می باشد
		
	\end{itemize}
\end{frame}
\begin{frame}
	
	\frametitle{ریسک}
	%\framesubtitle{subtitle}

	\begin{itemize}
		
		\item دو ویژگی ریسک 
			\begin{itemize}
			\item عدم قطعیت \lr{uncertainty}  :ممکن است روی بدهد یا ندهد یعنی احتمال وقوع حتمی ریسک وجود ندارد
			\item 		از دست دادن \lr{Loss} : اگر ریسک به واقعیت تبدیل شود پیامد و پیامدهای ناخواسته و یا مشکلاتی روی خواهد داد 
			 
		\end{itemize}
		\item در فرآیند تحلیل ریسک کمی سازی سطح عدم قطعیت و میزان مشکلات مرتبط با هر ریسک برای اولویت بندی ریسک از درجه اهمیت بالایی برخوردار است .(پرسمن ۲۰۰۹)

	\end{itemize}
\end{frame}
\begin{frame}
	\begin{figure}
		\centering
		\includegraphics[width=1.1\linewidth]{5}
		%\caption{}
		\label{fig:5}
	\end{figure}
	
\end{frame}
\begin{frame}
	\frametitle{انواع ریسک ها بر اساس ریشه}




	
	%\framesubtitle{subtitle}
	\begin{itemize}
		\item 	ریسک های داخلی :ریسکهایی که ریشه داخلی یا درون پروژه ای دارند
		\item 	ریسک های بیرونی : بعضی ریسکها ریشه خارجی دارند و با محیط اجرای پروژه مرتبط می‌باشند 
		\item 	ریسک های ترکیبی
	\end{itemize}
\end{frame}
\begin{frame}
	\frametitle{ریسک های خارجی}
	%\framesubtitle{subtitle}
	\begin{itemize}
		\item ریسک های خارجی بسیار متداول هستند :ویروس ها هکرها تغییر شرایط محیطی یا اقتصادی مشکلات مرتبط با مشتری و سازمان مشتری و سایر موارد اشاره کرد 
		\item صرفا مخصوص یک پروژه یا سازمان مشخص نیست و وقوع آنها تاثیرات فراگیری بر تعداد زیادی از سازمان ها یا پروژه ها خواهد داشت
		\item 	سازمان ها و پروژه ها توان محدودی در جلوگیری از وقوع این ریسک ها دارند 
		\item در مدیریت ریسک تمرکز باید بیشتر روی اجتناب، مدیریت و کاهش تاثیرات منفی آنها باشد
	\end{itemize}
\end{frame}
\begin{frame}
	\frametitle{ریسک های داخلی }
	%\framesubtitle{subtitle}
	\begin{itemize}
		\item عمدتاً ریشه در داخل سازمان یا پروژه دارند مثل استفاده نادرست از داده ها، مشکلات نرم‌افزاری ، اشتباهات پرسنلی و مشکلات مدیریتی 
		\item عمدتاً مختص سازمان یا پروژه خاصی است که به واسطه عملکرد یا شرایط خاص انجام پروژه ایجاد می گردد 
		\item با توجه به ریشه داخلی اینگونه ریسک ، مدیریت پروژه یا سازمان فعالیت‌های زیادی از جمله فعالیت‌های پیشگیرانه مدیریت و جذب تاثیرات منفی در رابطه با آنها می‌توانند انجام دهند 
		\item این ریسک هادرصد بالاتری از مخاطرات پروژه را تشکیل می‌دهند
		
	\end{itemize}
\end{frame}
\begin{frame}
	\frametitle{نمونه ریسک های بیرونی}
	%\framesubtitle{subtitle}
	 
	\begin{itemize}
		\item\textbf{تغییر در مدیریت مشتری} 
		\\
		برای اینگونه ریسک ها تیم پروژه نمی‌تواند کاری خاصی برای جلوگیری انجام دهد و برنامه ریزی باید روی مدیریت و کم کردن تاثیرات منفی متمرکز شود
			\\
		داشتن یک قرارداد صریح و مدون نیز می‌تواند به این امر کمک کند به این معنا که پروژه به نظرات مدیران کمتر وابسته باشد 
		\item\textbf{رکود بازار و مشکلات اقتصادی}  
		\\
		از این ریسک نیز نمی‌توان جلوگیری کرد و تمرکز باید روی آمادگی برای مدیریت در صورت وقوع باشد
	\end{itemize}
\end{frame}
\begin{frame}
	\frametitle{نمونه ریسک های داخلی}

	\begin{itemize}
		\item\textbf{عدم تحقق زمان‌بندی} \\
		علت: ناشناخته بودن شرایط، برنامه ریزی غلط یا اشتباهات مدیریتی \\
		مدیریت ریسک  : تجزیه و تحلیل اثرات منفی تاخیرات-جلوگیری از تاخیرات-کم کردن اثرات منفی تاخیرات در صورت وقوع
		
		\item\textbf{رد خروجی‌ها از جانب مشتری} \\
		علت: عدم ارتباط صحیح بین مشتری و مجری انتظارات نابجا و ناآگاهی مشتری 
		\\مدیریت ریسک : بیشتر روی جلوگیری تمرکز دارد به عبارت دیگر جلوگیری از به وقوع پیوستن -	بهترین فعالیت جلوگیری :ارتباط موثر بین تیم پروژه و مشتری -تعریف درست خروجی‌ها ،گستره و معیارهای پذیرش 
		
		\item\textbf{هزینه کردن بیش از برنامه و مشکلات مالی مربوطه}\\ 
		
		مدیریت ریسک : برای مقابله با این ریسک تمرکز باید روی جلوگیری( با برنامه ریزی درست و نیز اجرای با دقت مدیریت هزینه در پروژه) باشد
		
		\item\textbf{خروج افراد کلیدی از پروژه  } \\
	مدیریت ریسک : سیاست جلوگیری از رفتن پرسنل کلیدی و تبادل اطلاعات به صورتی که اطلاعات کلیدی فقط در اختیار افراد محدود نباشد از روش‌های مقابله با این ریسک می باشد
		
	\end{itemize}
\end{frame}
\begin{frame}
	\frametitle{نمونه ریسک های ترکیبی}
	%\framesubtitle{subtitle}
	 
	
	
	
	\begin{itemize}
		\item\textbf{تاخیر عامل سوم در ارائه خروجی های مربوطه} \\
		فعالیت عامل سوم منشاء بیرونی دارد ولی مجری در انتخاب عامل سوم تا حدی آزادی عمل دارد 
		\\مدیریت ریسک : با انجام  فعالیت های داخلی مثل گنجاندن جریمه های قابل توجه برای تاخیر د و داشتن برنامه های جایگزین و اضطراری\\
		\item\textbf{عدم آمادگی مشتری در پذیرش و پیاده سازی } \\
		
		 مشکلات عمده : تاخیر در پرداخت نهایی به مجری - قفل شدن نیروها در یک پروژه خاص با سربار های مالی مربوطه - احتمال وقوع اختلافات\\
		این ریسک هم جنبه بیرونی داشته زیرا آمادگی مشتری یک مورد درون‌سازمانی مشتری است و هم دست مجری برای اطمینان از این آمادگی نیز باز است (ریسک ترکیبی)\\
		مدیریت ریسک : تمرکز باید روی جلوگیری باشد: مشخص کردن و ارتباط پیش فرض ها و انتظارات و کارهایی که باید توسط مشتری انجام گیرد 
	\end{itemize}
\end{frame}
\begin{frame}
	\frametitle{استراتژی های مقابله با ریسک }
	3 استراتژی اصلی با توجه به نوع تهدید، تأثیرات ریسک ،احتمال وقوع ریسک و شرایط برای مدیریت ریسک :\\
	
	\centering
			\includegraphics[width=1\linewidth]{۱۲}


	
\end{frame}
\begin{frame}
	\frametitle{استراتژی های مقابله با ریسک }
	استراتژی مقابله یا مدیریت ریسک را می‌توان به استراتژی‌های واکنشی یا پیشگیرانه تقسیم کرد (پرسمن ۲۰۰۹)
	\\
	\begin{center}
		\includegraphics[width=1\linewidth]{13}
	\end{center}
	
\end{frame}

\end{document}